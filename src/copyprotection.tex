\chapter{copyright atari}
\label{sec:copyprotection}
\lhead[tempest]{}
\lstset{style=6502Style}

\section*{\icode{QT1}: Copyright Atari}
\begin{lstlisting}
ZATLIS: ASCVH <^ MCMLXXX ATARI>
\end{lstlisting}

\begin{lstlisting}
ZATLIV: CLC           ; CLEAR THE CARRY BIT.
        LDY I,ZATLIC  ; GET POS OF LAST CHAR IN STRING.
        LDA I,085     ; INITIALIZE QT1.
        BEGIN         ; LOOP THROUGH COPYRIGHT STRING.
        ADC Y,ZATLIS  ; ADD VALUE OF CURRENT CHARACTER.
        DEY           ; GO TO NEXT CHAR IN STRING.
        MIEND         ; KEEP LOOPING UNTIL END.
        STA QT1       ; VERIFY ATARI LITERAL
        RTS
\end{lstlisting}

The punishment: make the game impossibly fast.
\begin{lstlisting}
ZQATLI: LDA QT1       ; GET QT1.
        IFNE          ; IF IT'S NON ZERO..
        LDA WAVEN1    ; CHECK THE CURRENT LEVEL..
        CMP I,10.     ; IF IT'S GREATER THAN 10..
        IFCS          ; THEN..
        LDA I,7A      ; SET AN IMPOSSIBLE TO PLAY
        STA FRTIMR    ; FRAMERATE.
        ENDIF
        ENDIF
\end{lstlisting}

\section*{\icode{QT2}: Copyright Atari}
Check that the Atari credit message is being displayed. This checks that
the code at \icode{ZATC4S}, which displays the Atari copyright message,
has not been modified.

\begin{lstlisting}
ZATC4S: LDX I,MATARI  ; GET THE ATARI (C) MESSAGE.
        JSR MSGS      ; DISPLAY IT.
        LDX I,MCREDI  ; GET THE 'CREDITS' MESSAGE.
        JSR MSGS		  ; DISPLAY IT.
ZATC4E:
ZATC4C==ZATC4E-ZATC4S
\end{lstlisting}
Notice how we identify the end of the code block by adding a label \icode{ZATC4E}
and then used that and the label at the start of the block (\icode{ZATC4S}) to calculate
the length of the code block and store it as \icode{ZATC4C}.
This comes in handy when we want to iterate through
the code at runtime and verify that it hasn't been altered:
\begin{lstlisting}
ZATC4V: LDY I,ZATC4C		; GET THE LENGTH OF THE CODEBLOCK.
        LDA I,0A7       ; INITIALIZE OUR CHECKSUM. 
        BEGIN           ; LOOP THROUGH EACH BYTE IN THE CODEBLOCK.
        EOR Y,ZATC4S    ; XOR VAL OF CURRENT BYTE WITH CHECKSUM.
        DEY             ; GO TO NEXT BYTE
        MIEND           ; LOOP THROUGH FULL CODE BLOCK.
        STA QT2         ; STORE CHECKSUM IN Q2.
\end{lstlisting}

\section*{\icode{QT3}: The Display List}
This one checks the contents of the vector display list used to render
the Atari copyright message (as well as everything else the game writes to the screen).
Since the copyright message is displayed on a few different
screens, with different colors, the contents of the display list can vary,
even when the Atari copyright banner is correctly displayed, so our checksum
has to tolerate a bit of variety.

Every time a message is written, the \icode{MSGS} routine checks if the
copyright message has been passed to it: if so, it stores the current
location in the display list in \icode{SECUVG} for checking later.

\begin{lstlisting}
ZSECL0: CPX I,MATARI
        IFEQ                    ;COPYRIGHT MSG?
        LDA VGLIST              ;YES. SAVE START LOC
        STA SECUVG
        LDA VGLIST+1
        STA SECUVG+1
        ENDIF
\end{lstlisting}

When the \icode{DISPLAY} routine runs, we are processing the contents
of the display list contained in \icode{VGLIST}. It's time for us to check
the portion that contains our hallowed copyright message. 
\begin{lstlisting}
ZATVG2::
  LDA SECUVY      ; ARE WE DISPLAYING ONE OF THE TITLE SCREENS?
  IFNE            ; IF WE ARE THEN:
  LDY I,27        ; FOR ALL 27 BYTES IN THE ROW.
  LDA I,0E        ; INITIALIZE OUR CHECKSUM VALUE
  SEC             ; CLEAR THE CARRY BIT SO IT DOESN'T INTERFERE
  ; LOOP THROUGH EACH CHARACTER IN THE LINE POINTED TO BY SECUVG
  ; THIS SHOULD CONTAIN '(C) MCMLXXX ATARI', OTHERWISE WE WILL GET
  ; AN INVALID CHECKSUM.
  BEGIN           ; LOOP FROM 27 TO 0
  SBC NY,SECUVG   ; SUBTRACT VALUE IN THE CURRENT CHAR FROM 'A' REGISTER.
  DEY             ; DECREMENT Y TO GO TO THE PREVIOUS CHARACTER.
  MIEND           ; LOOP UNTIL Y IS 0.
  TAY             ; STORE THE RESULT IN A IN Y.
  IFNE            ; IF THE RESULT IS ZERO, THE CHECKSUM PASSES OTHERWISE:
  EOR I,0E5       ; CHECK IT AGAINST THE CHECKSUM FOR ANOTHER SCREEN
  ENDIF           
  IFNE            ; IF THAT PASSES WE'RE DONE OTHERWISE:
  EOR I,02A       ; CHECK IT AGAINST THE CHECKSUM FOR ANOTHER SCREEN
  ENDIF           
  STA QT3         ; STORE THE RESULT OF OUR CHECK IN QT3.
  ENDIF
  ENDIF
\end{lstlisting}

\begin{lstlisting}
ZQVAVG::
  LDA QT3     ; CHECK THAT BOTH QT3 AND
  ORA QT6     ; QT6 ARE ZERO.
  IFNE        ; IF THEY ARE NOT THEN
  LDA I,17    ; CHECK IF THE PLAYER' SCORE IS GREATER THAN 170,000
  CMP LSCORH  ; LSCORH CONTAINS THE FIRST 2 DIGITS OF THE PLAYER SCORE
  IFCC        ; IF IT IS GREATER THAN OR EQUAL TO 17..
  LDX LSCORL  ; LOAD WHATEVER IS IN THE LSCORL BYTE
  INC X,0     ; AND USE THAT TO INCREMENT ONE OF OUR 'ZERO-PAGE' BYTES.
  ENDIF       ; IN THE HOPE OF CAUSING SOME HAVOC.
  ENDIF       ; HAVOC SECURED.
\end{lstlisting}

\section*{\icode{QT4}: Sound}
This tests the sound chips (known as 'pokeys') for expected behaviour. There are two sound chips on the 
board and each has a hardware-based random number generator. Retrieving a random number
from each, one after the other, should return matching values. If they don't, something is wrong.
The check is looking out for signs of the pokeys temporarily stopping when queried for a random
number. Don't know why this is bad.

\begin{lstlisting}
ZPOKST: LDX I,4     ; QUERY THE POKEYS 4 TIMES.
        LDA RANDOM  ; GET A RANDOM NUMBER FROM POKEY 1.
        LDY RANDO2  ; GET A RANDOM NUMBER FROM POKEY 2.
        BEGIN       ; LOOP AND COMPARE SUBSEQUENT VALUES 4 TIMES.
        CMP RANDOM  ; GET A NEW VALUE FROM POKEY 1 AND COMPARE TO ORIGINAL.
        IFEQ        ; IF IT MATCHES..
        CPY RANDO2  ; GET A NEW VALUE FROM POKEY 2 AND COMPARE TO ORIGINAL.
        ENDIF
        IFNE        ; IF NEITHER MATCHED, THEN STORE THE FAILURE..
        STA QT4     ; IN OUR QT4 CHECKSUM VALUE.
        LDX I,0     ; SKIP TO END BY SETTING X TO 0.
        ENDIF
        DEX         ; DECREMENT X
        MIEND       ; LOOP UNTIL X IS -1
\end{lstlisting}

If this check fails, then we are going to ruin the player's day by corrupting
the computer's stack. We do this in the \icode{DSPEXP} routine, so as long as you
never hit anything on your bootlegged version of \textit{Tempest}, you should be OK.

\begin{lstlisting}
ZQPOKS: LDA QT4     ; CHECK THE SOUND CHIP CHECKSUM.
        IFNE        ; IF IT FAILED THEN SET UP A BOMB..
        LDA CURWAV  ; GET THE CURRENT LEVEL..
        CMP I,13.   ; COMPARE IT TO 13.
        IFCS        ; IF IT IS GREATER THAN 13..
        STA 1FF     ; KILL TOP OF STACK
        ENDIF
        ENDIF
        RTS
\end{lstlisting}

\section*{\icode{QT5}: Sound}

\begin{lstlisting}
        .SBTTL  DISPLAY HIGH SCORE LADDER
LDRDSP: JSR INFO                ;DISPLAY SCORE & LIVES INFO
ZPONTS: SEI          ; DISABLE INTERRUPTS.
        LDA RANDOM
        LDY RANDOM
        STY TEMP0
        LSR
        LSR
        LSR
        LSR
        EOR TEMP0
        STA TEMP0
        LDA RANDO2
        LDY RANDO2
        CLI         ; ALLOW INTERRUPTS AGAIN.
        EOR TEMP0
        AND I,0F0
        EOR TEMP0
        STA TEMP0
        TYA
        ASL
        ASL
        ASL
        ASL
        EOR TEMP0
        STA QT5     ; STORE THE RESULT IN QT5.
\end{lstlisting}

Randomly tagged on to the end of the display starfield routine (\icode{DSTARF}). 
If the score is greater than 150,000 then use the last few digits of the score
to randomly increment a byte in memory at position 200 or above. Like the damage
wrought in \icode{QT3} this will be done every time the starfield display is updated
during a level change so will inevitably, and promptly, bring the game to a crashing
halt.

\begin{lstlisting}
ZQPONS: LDA QT5
        IFNE
        LDX LSCORH
        CPX I,15
        IFCS
        LDX LSCORL
        INC X,200
        ENDIF
        ENDIF
        RTS
\end{lstlisting}

\section*{\icode{QT6}: Atari Again!}
This is the same check as \icode{QT3} but is run in attract mode. The display list
is inspected and a checksum calculated on the line where the Atari copyright message
is expected to be.
\begin{lstlisting}
ZATVG1: LDA QSTATUS      ; ARE WE IN ATTRACT MODE?
        IFPL             ; IF YES THEN.. ATARI BETTER BE ON SCREEN
        LDA I,0F2        ; INITIALIZE THE CHECKSUM VALUE TO F2.
        CLC              ; CLEAR THE CARRY BIT.
        LDY I,39.        ; ITERATE OVER ALL 39 BYTES ON THE LINE
        BEGIN            ; LOOP THROUGH ALL 39 BYTES..
        ADC NY,SECUVG    ; ADDING THE VALUE AT EACH BYTE TO OUR CHECKSUM
        DEY              ; KEEP LOOPING FOR ALL 39.
        MIEND            ; END LOOP
        STA QT6          ; SAVE THE RESULT (WHICH SHOULD BE 0)
        ENDIF
\end{lstlisting}

Like \icode{QT3}, an invalid checksum in \icode{QT3}, will cause rapid corruption
of bytes in the zero-page area of the machine's memory. 
\begin{lstlisting}
ZQVAVG::
  LDA QT3     ; CHECK THAT BOTH QT3 AND
  ORA QT6     ; QT6 ARE ZERO.
  IFNE        ; IF THEY ARE NOT THEN
  LDA I,17    ; CHECK IF THE PLAYER' SCORE IS GREATER THAN 170,000
  CMP LSCORH  ; LSCORH CONTAINS THE FIRST 2 DIGITS OF THE PLAYER SCORE
  IFCC        ; IF IT IS GREATER THAN OR EQUAL TO 17..
  LDX LSCORL  ; LOAD WHATEVER IS IN THE LSCORL BYTE
  INC X,0     ; AND USE THAT TO INCREMENT ONE OF OUR 'ZERO-PAGE' BYTES.
  ENDIF       ; IN THE HOPE OF CAUSING SOME HAVOC.
  ENDIF       ; HAVOC SECURED.
\end{lstlisting}

