\chapter{tanker detail}
\label{sec:attract_mode}
\lhead[tempest]{}
\lstset{style=6502Style}

A standard tanker in \textit{Tempest} releases flippers when it is struck.

\begin{minipage}[c]{0.52\linewidth}
\begin{figure}[H]
    \centering
    \begin{adjustbox}{width=6.75cm,center}
      \includegraphics[width=12cm]{src/tankers/vec_image_plot_tankr.png}%
    \end{adjustbox}
\end{figure}
\end{minipage}
\hspace{0.5cm}
\begin{minipage}[c]{0.40\linewidth}
\begin{lstlisting}[basicstyle=\scriptsize\ttfamily]
TANKR:
        ICVEC
        SCVEC 20,0,0
GENTNK: CSTAT PURPLE
        SCVEC 0,20,CB
        SCVEC 0,0C,CB
        SCVEC 20,0,CB
        SCVEC 0C,0,CB
        SCVEC 0,0C,CB
        SCVEC -0C,0,CB
        SCVEC 0,20,CB
        SCVEC -20,0,CB
        SCVEC -0C,0,CB
        SCVEC 0,-0C,CB
        SCVEC -20,0,CB
        SCVEC 0,-20,CB
        SCVEC 0,-0C,CB
        SCVEC 0C,0,CB
        SCVEC 0,-20,CB
        SCVEC 20,0,CB
        SCVEC 0C,0,CB
        RTSL
\end{lstlisting}
\vspace*{\fill}
\end{minipage}

A detail that is almost hidden, or at least very easy to miss, is that the tankers containing
pulsars and fuses contain a rubric indicating their contents.  

The tanker containing fuses contains a very basic representation of the fuse enemy.
\begin{minipage}[c]{0.67\linewidth}
\begin{figure}[H]
    \centering
    \begin{adjustbox}{width=9.6cm,center}
      \includegraphics[width=12cm]{src/tankers/vec_image_plot_tankf.png}%
    \end{adjustbox}
\end{figure}
\end{minipage}
\hspace{0.5cm}
\begin{minipage}[c]{0.25\linewidth}
\begin{lstlisting}[basicstyle=\scriptsize\ttfamily]
TANKF: ICVEC
      CSTAT BLUE
      SCVEC -0C,0,CB
      SCVEC 0,0C,0
      CSTAT RED
      SCVEC 0,0,CB
      CSTAT GREEN
      SCVEC 0,-0C,CB
      SCVEC 0,0,0
      CSTAT YELLOW
      SCVEC 0C,0,CB
      SCVEC 20,0,0
GENTNK:
      CSTAT PURPLE
      SCVEC 0,20,CB
      SCVEC 0,0C,CB
      SCVEC 20,0,CB
      SCVEC 0C,0,CB
      SCVEC 0,0C,CB
      SCVEC -0C,0,CB
      SCVEC 0,20,CB
      SCVEC -20,0,CB
      SCVEC -0C,0,CB
      SCVEC 0,-0C,CB
      SCVEC -20,0,CB
      SCVEC 0,-20,CB
      SCVEC 0,-0C,CB
      SCVEC 0C,0,CB
      SCVEC 0,-20,CB
      SCVEC 20,0,CB
      SCVEC 0C,0,CB
      RTSL
\end{lstlisting}
\vspace*{\fill}
\end{minipage}

The tanker for \textit{pulses}, given opposite, contains a more successful effort. It
is much more immediately legible during play, compared to the cross on a \textit{fuse}
tanker which gets slightly lost in the detail.

Note the relative economy of coding used to draw each tanker. The rubrics are drawn
by the first few lines, i.e. \icode{TANKF} and \icode{TANKP}. In each case the remaining
section (\icode{GENTNK}) is common to all tanker types. I've given \icode{GENTNK} in 
full in each of the diagrams for the purposes of clarity - but in practice the duplication
of code was avoided by replacing each with a \icode{JMP} call to the \icode{GENTNK} routine, which
only had to be written once. For example, in the case of the pulsar tank:
\begin{lstlisting}
TANKP:ICVEC
      CSTAT TURQOI
      SCVEC -5,-2,0
      SCVEC -3,6,CB
      SCVEC 0,-6,CB
      SCVEC 3,6,CB
      SCVEC 5,-2,CB
      SCVEC 20,0,0
      JMP GENTNK
\end{lstlisting}


\begin{minipage}[c]{0.67\linewidth}
\vspace{-0.5cm}
\begin{figure}[H]
    \centering
    \begin{adjustbox}{width=8.5cm,center}
      \includegraphics[width=12cm]{src/tankers/vec_image_plot_tankp.png}%
    \end{adjustbox}
\end{figure}
\end{minipage}
\hspace{0.5cm}
\begin{minipage}[c]{0.25\linewidth}
\vspace{-0.5cm}
\begin{lstlisting}[basicstyle=\scriptsize\ttfamily]
TANKP:ICVEC
      CSTAT TURQOI
      SCVEC -5,-2,0
      SCVEC -3,6,CB
      SCVEC 0,-6,CB
      SCVEC 3,6,CB
      SCVEC 5,-2,CB
      SCVEC 20,0,0
GENTNK:
      CSTAT PURPLE
      SCVEC 0,20,CB
      SCVEC 0,0C,CB
      SCVEC 20,0,CB
      SCVEC 0C,0,CB
      SCVEC 0,0C,CB
      SCVEC -0C,0,CB
      SCVEC 0,20,CB
      SCVEC -20,0,CB
      SCVEC -0C,0,CB
      SCVEC 0,-0C,CB
      SCVEC -20,0,CB
      SCVEC 0,-20,CB
      SCVEC 0,-0C,CB
      SCVEC 0C,0,CB
      SCVEC 0,-20,CB
      SCVEC 20,0,CB
      SCVEC 0C,0,CB
      RTSL
\end{lstlisting}
\vspace*{\fill}
\end{minipage}
