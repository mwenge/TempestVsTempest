\chapter{strings of tempest 2000}
\lhead[tempest 2000]{}
\label{sec:bullet}
\lstset{style=68KStyle}
Three different fonts are used in \textit{Tempest 2000}. Each one is
set out in a sprite sheet, and comes with an accompanying array that
maps ASCII characters a set of co-ordinates in the sheet. 

\subsection*{afont - the 'Atari' font}
For example in the \icode{afont}, also known as the \textit{Atari} font, the
letter \icode{A} is found at X position 1 and Y position 164 (\icode{\$A4}),
while \icode{B} is at X position 20 (\icode{\$14}) and Y position 164 (\icode{\$A4}), 
in the sprite sheet \icode{beasty4.cry}.

\begin{minipage}[c]{0.58\linewidth}
\begin{figure}[H]
    \centering
    \begin{adjustbox}{width=8.6cm,center}
      \includegraphics[width=12cm]{src/cry/beasty4.png}%
    \end{adjustbox}
\end{figure}
\end{minipage}
\hspace{0.5cm}
\begin{minipage}[c]{0.38\linewidth}
\begin{lstlisting}[basicstyle=\scriptsize\ttfamily]
afont:  
  dc.l pic2      ; beasty4.cry
  dc.l $0012000f ; 
  dc.l $b6011e   ; Space
  dc.l $a40001   ; A
  dc.l $a40014   ; B
  dc.l $a40027   ; C
  dc.l $a4003a   ; D
  dc.l $a4004d   ; E
  dc.l $a40060   ; F
  dc.l $a40073   ; G
  dc.l $a40086   ; H
  dc.l $a40099   ; I
  dc.l $a400ac   ; J
  dc.l $a400bf   ; K
  dc.l $a400d2   ; L
  dc.l $a400e5   ; M
  dc.l $a400f8   ; N
\end{lstlisting}
\vspace*{\fill}
\end{minipage}

What this looks like in practice is demonstrated below. The copyright message
from the title screen is displayed by loading the \icode{afont}, the text string,
and then using the \icode{centext} routine to render the string to the display.
\begin{figure}[H]
    \centering
    \begin{adjustbox}{width=13cm,center}
      \includegraphics[width=12cm]{src/t2k_strings/copyright_1981,.png}%
    \end{adjustbox}
    \begin{adjustbox}{width=13cm,center}
      \includegraphics[width=12cm]{src/t2k_strings/1994_Atari_Corp..png}%
    \end{adjustbox}
\end{figure}
\vspace{-0.6cm}
\begin{lstlisting}
; Write "Copyright 1981" to screen.
lea afont,a1     ; Select afont image map for our font.
lea ataricop1,a0 ; Load "Copyright 1981" string to a0
move #190-8,d0   ; Set the Y Pos.
add palfix2,d0   ; Adjust for PAL.
jsr centext      ; Write the text in a0 to the screen.

; Write "1994 Atari Corp" to screen.
lea afont,a1     ; Select afont image map for our font.
lea ataricop2,a0 ; Load "1994 Atari Corp" string to a0.
move #207-5,d0   ; Set the Y pos of the text.
add palfix2,d0   ; Adjust for PAL.
jsr centext      ; Write the text in a0 to the screen.
\end{lstlisting}

The routine \icode{centext} is useful for display static, horizontally
centred messages. But what about cases where we want to display a message
for a brief period of time? For this we have a pair of routines called
\icode{setmsg} and \icode{drawmsg} that are used to display contextual
messages to the player in the 'Atari' font. For example, when they are
doing especially well and get a 'power up':
\begin{figure}[H]
    \centering
    \begin{adjustbox}{width=13cm,center}
      \includegraphics[width=12cm]{src/t2k_strings/yes!_yes!_yes!.png}%
    \end{adjustbox}
\end{figure}
\vspace{-0.6cm}
\begin{lstlisting}
lea drtxt,a0               ; Point a0 to  "yes! yes! yes!" string.
clr.l d0                   ; Set X position.
move.l #$8000,d1           ; Set Y position.
bsr setmsg                 ; Display the message.
\end{lstlisting}

Once we've defined the message and its horizontal and vertical position above, we
call \icode{setmsg} to put it on a display timer. We also set its initial X and
Y position as well as the final position of the message as it 'decays'.

Every time we draw objects on the screen we will call the \icode{drawmsg} routine
to render any active messages in \icode{afont}. We don't give that routine in full
here as it is quite long!
\clearpage
\begin{lstlisting}
; *******************************************************************
; Update the message displayed to the player with whatever is in a0.
; The display of the message is handled by 'drawmsg'.
; *******************************************************************
setmsg:
        move #100,msgtim1          ; set default Messager parameters
        move #50,msgtim2
        move.l d0,msgxv            ; X position for second phase of message.
        move.l d1,msgyv            ; Y position for second phase of message.
        move.l #$10000,msgxs       ; Initial X position.
        move.l #$10000,msgys       ; Initial Y position.
        move.l a0,msg              ; Set the message object.
        rts
\end{lstlisting}


\subsection*{bfont - the 'Paused' font}
This metallic space-opera of a font is used primarily for the 'Demo' and 'Paused'
messages but is also used when entering a high score.

\begin{minipage}[c]{0.58\linewidth}
\begin{figure}[H]
    \centering
    \begin{adjustbox}{width=8.6cm,center}
      \includegraphics[width=12cm]{src/cry/beasty8.png}%
    \end{adjustbox}
\end{figure}
\end{minipage}
\hspace{0.5cm}
\begin{minipage}[c]{0.38\linewidth}
\begin{lstlisting}[basicstyle=\scriptsize\ttfamily]
bfont:
  dc.l pic6     ; beasty8.cry
  dc.l $001f001b
  dc.l $470119  ;Space
  dc.l $010001  ;A
  dc.l $010024  ;B
  dc.l $010047  ;C
  dc.l $01006a  ;D
  dc.l $01008d  ;E
  dc.l $0100b0  ;F
  dc.l $0100d3  ;G
  dc.l $0100f6  ;H
  dc.l $010119  ;I
  dc.l $240001  ;J
  dc.l $240024  ;K
  dc.l $240047  ;L
  dc.l $24006a  ;M
  dc.l $24008d  ;N
\end{lstlisting}
\vspace*{\fill}
\end{minipage}

Here's how it looks during the 'Pause' screen.
\begin{figure}[H]
    \centering
    \begin{adjustbox}{width=7cm,center}
      \includegraphics[width=12cm]{src/t2k_strings/paused.png}%
    \end{adjustbox}
\end{figure}
\vspace{-0.6cm}
\begin{lstlisting}
jsr text2_setup
; Display the 'paused' text.
lea in_buf,a0           ; Point our GPU RAM input buffer at a0.
move.l #$490074,32(a0)  ; was 35
move.l #pautext,(a0)    ; Point a0 at "paused" string.
lea texter,a0           ; Load the GPU module in stoat.gas.
jsr gpurun              ; Run the selected GPU module.
jsr gpuwait             ; Wait for the GPU to finish.
\end{lstlisting}

The 'Pause' screen has a commented-out feature that was likely in use when the
game was going through testing and review. This is a date-stamped version string,
referencing the date of the current build. What we find in the source code is the
date of the last build that this feature was used: 12th November 1993, or 11th
December 1993. My guess is the latter - since the intended audience for the test
builds were American.
\begin{figure}[H]
    \centering
    \begin{adjustbox}{width=13cm,center}
      \includegraphics[width=14cm]{src/t2k_strings/t2k_version_121193.png}%
    \end{adjustbox}
\end{figure}
\vspace{-0.6cm}
\begin{lstlisting}
jsr text2_setup ; Select 'bfont'.
move.l #$400040,32(a0) ; Set position.
move.l #vertext,(a0) ; Select version string.
lea texter,a0   ; Load GPU module in stoat.gas.
jsr gpurun      ; Run selected GPU module.
jsr gpuwait     ; Wait for GPU to finish.
\end{lstlisting}

\subsection*{cfont - the 'Arcade' font}
This one has the unusual distinction of containing two fonts in one: or rather
the same font in different colours. Upper case strings are mapped to a green/orange
version of the font, while lower case strings are mapped to a blue/white version.
\begin{minipage}[c]{0.58\linewidth}
\begin{figure}[H]
    \centering
    \begin{adjustbox}{width=8.6cm,center}
      \includegraphics[width=12cm]{src/cry/beasty3.png}%
    \end{adjustbox}
\end{figure}
\end{minipage}
\hspace{0.5cm}
\begin{minipage}[c]{0.38\linewidth}
\begin{lstlisting}[basicstyle=\scriptsize\ttfamily]
cfont:
  dc.l pic      ; beasty3.cry
  dc.l $00080008
  dc.l $b50137  ;Space
  dc.l $b5001b  ;A
  dc.l $b50026  ;B
  dc.l $b50031  ;C
  dc.l $b5003c  ;D
  dc.l $b50047  ;E
  dc.l $b50052  ;F
  dc.l $b5005d  ;G
  dc.l $bf001b  ;a
  dc.l $bf0026  ;b
  dc.l $bf0031  ;c
  dc.l $bf003c  ;d
  dc.l $bf0047  ;e
  dc.l $bf0052  ;f
  dc.l $bf005d  ;g
\end{lstlisting}
\vspace*{\fill}
\end{minipage}

We can see an example of this in the string \textit{"press FIRE to play"}. \textit{"FIRE"}
is rendered in the green/orange font, while the rest of the text is rendered in the blue/white
font used for lower case text.

\begin{figure}[H]
    \centering
    \begin{adjustbox}{width=13cm,center}
      \includegraphics[width=12cm]{src/t2kdemo/pressfiretoplay.png}%
    \end{adjustbox}
\end{figure}
\vspace{-0.6cm}
\begin{lstlisting}
; Draw 'press FIRE to play'
lea cfont,a1   ; Load the small font
lea autom2,a0  ; "press FIRE to play"
move #180,d0   ; Set y position
jsr centext    ; Draw the text in the centre.
\end{lstlisting}


\begin{figure}[H]
    \centering
    \begin{adjustbox}{width=13cm,center}
      \includegraphics[width=12cm]{src/t2k_strings/enter_something_EGOTISTICAL!.png}%
    \end{adjustbox}
\end{figure}
\vspace{-0.6cm}
\begin{lstlisting}
lea cfont,a1         ; Load the small regular font to a1.
move.l enl3,a0       ; "enter something EGOTISTICAL!"
move #80,d0          ; Set Y position of text.
jsr centext          ; Display horizontally centred text.
\end{lstlisting}

\begin{lstlisting}
; *******************************************************************
; text_setup
; Set up text in the 'Atari' font.
; *******************************************************************
text_setup:
        lea in_buf,a0              ; Point our GPU RAM input buffer at a0.
        move.l #afont,4(a0)        ; 'Atari' font data structure
tsu:
        move.l #0,8(a0)            ; Start of font data structure.
        move.l #0,12(a0)           ; Drop-shadow in X direction.
        move.l #$10000,d0          ; Store 10000 in d0.
        move.l d0,16(a0)           ; Drop-shadow in Y direction.
        move.l d0,20(a0)           ; Scale factor in X direction.
        move.l #0,24(a0)           ; Scale factor in Y direction.
        move.l #0,28(a0)           ; Text shear in X direction.
        move.l #0,36(a0)           ; Mode 0: Straight Draw.
        rts
\end{lstlisting}

